%discussion
\documentclass[../../main/main.tex]{subfiles}

\begin{document}
\title{Discussion}

%%%%%%%%%%%%%%%%%%%%% Chapter Discussion %%%%%%%%%%%%%%%
\chapter{Discussion}\label{chp:discussion}

      %%%%%%%%%%%%%%%%%%% Section Recap %%%%%%%%%%%%%%%%%%
\section{Recap}\label{recap}

      %%%%%%%%%%%%%%%%%%% Section Mission Accomplished %%%%%%%%%%
\section{Mission Accomplished}\label{missionaccomplished}

      %%%%%%%%%%%%%%%%%%% Section Stop-gap %%%%%%%%%%%%%%%%
\section{Stop-Gaps, Lessons Learned, \& Advice}\label{stopgap}

      %%%%%%%%%%%%%%%%%%% Section Other Verifiable Theories %%%%%%%%
\section{Other Verifiable Theories}\label{otherTheories}
There were other modules and complexities in the patrol base operations that were discussed during the course of the work.  However, because of time constraints they were no verified in the \glsentryshort{acl} and \glsentryshort{hol}.

\subsection{Authentication: Roles and Cypto}
This master thesis discussed authentication as a visual confirmation among the soldiers. For example, most often, most soldiers in a platoon know who their platoon leader is and do not require a password.  But, this is not always the case.  One example where it may not be readily obvious is when the platoon leader has been incapacitated and can no longer function in that role.  The military has a clearly defined succession hierarchy and soldier are (or should be) aware of this.  But, computers aren't as smart as soldiers...unless they are designed to be so.  In this case, the concept of soldiers acting in roles is useful.

The idea behind defining the platoon leader as the authorized principal in, for example, the top level \glsentryshort{ssm} allows for the system to work regardless of who the platoon leader is.  Nevertheless, it is the soldier acting in the role of the platoon leader who is actually authorized.  This can also be represented in the \glsentryshort{ssm} model.  In fact, a parametrizable \glsentryshort{ssm} is already designed and implemented in \glsentryshort{hol} to do this.

In this model, the \textit{reps} rule from figure \ref{inferencerules} does this.  


\subsection{Soldier Theory}
\subsection{Platoon Theory}
\subsection{Squad And Fire Team Theories}




\end{document}