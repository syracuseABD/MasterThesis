%introduction
\documentclass[../../main/main.tex]{subfiles}

\pagenumbering{arabic}

\begin{document}
\glsresetall

\chapter{Introduction}
 Imagine this scene from the movie titled "The Kingdom" \cite{kingdom}.  It is a bright, sunny day in Saudi Arabia.  The compound is isolated from the rest of the Kingdom and heavily guarded by Saudi police.  The American civilian contractors are free to enjoy the American culture they are accustomed to without offending the locals.  A large group of contractors and their families are enjoying their day off watching a game of little league on the compound.  
 
 
Unexpectedly, two men dressed in Saudi police uniforms begin showering the crowd with machine gun fire.   Civilian men, women, and children run screaming for cover. The two gun men are eventually neutralized by Saudi police officers.  Saudi police yell to the survivors "come this way" to lure them away from the carnage. One of these men dressed in a police uniform shouts "Allah Akbar" and detonates, killing himself and several others near him. 

What went wrong?  How did these terrorists gain access to a secured compound?  

Movies like this and news stories reporting real terrorist attacks on civilians are bringing the reality of security to the forefront of everyone's mind.   Everyone is concerned with controlling access to their information, their property, and more importantly their lives.   

To drive it home, consider the importance of security in computer driven cars, bank accounts, medical records, pace makers, etc.  Security can be as simple as providing your daughter with a secret password for who can pick her up from school or as complicated as managing access to national security secrets.  In any and every system that is vulnerable to compromise or attack, security is one of the foremost considerations. 


This chapter introduces the reader to a methodology for designing secure systems.  It briefly touches upon how this methodology fits in with current standards in systems engineering and systems security engineering. 

%  It then introduces an approach to verifying and documenting this property using the Certified Security by Design (\glsentryshort{csbd}) approach.  After introducing the \glsentryshort{csbd}, this chapter presents an application of \glsentryshort{csbd} that aims to close the gap in its field of applicability.  It discuss how \glsentryshort{csbd} is applied to a non-automated, human-centered military system (patrol base operations).  Finally, this chapter offers the reader a glimpse at what to expect in the subsequent chapters.

      %%%%%%%%%%%%%%%%%%%% Section csbd %%%%%%%%%%%%%%%%%%%%%%
\section{In Context of Systems Security Engineering}\label{sec:intro:motivation}
The authority on standards in systems engineering is the International Organization for Standardization (ISO).  In one of its published standards (ISO/IEC/IEEE 15288 \cite{iso15288}) it describes systems engineering guidelines for managing man-made systems.  These man-made systems span the range from human-centered systems to fully automated systems.  An example of a human-centered system is a partially- or non-automated system for controlling access to a base or compound.  An example of a fully automated system is the computer that controls a driverless car.  In all these systems, security is often a critical component of the systems engineering process.


The recognized authority in systems security engineering, a subspecialty of systems engineering, is the National Institute for Standards and Technology (NIST) Special Publication 800-160 (vol 1 and vol 2) \cite{NIST800160}.  It describes a framework for designing trustworthy systems by engineering security into the system from the start.  


This master thesis discusses the \Gls{storm} methodology for designing secure systems.  \gls{storm} applies state-of-the-field practices to design systems that are consistent with ISO/IEC/IEEE standards and the NIST Systems Security Engineering Framework.  These standards ensure that security is the forefront  consideration of all security-sensitive systems.  
%
%A subspecialty of systems security engineering manages access to security-sensitive objects using the principle of complete mediation.  Complete mediation broadly refers to verifying and documenting authentication and authorization of access to security sensitive objects. 

\section{System-Theoretic Operational Risk Management (STORM)}\label{sec:intro:storm}
\Gls{storm} is comprised of two components that each focus on some aspect of the Systems Security Engineering Framework.  The first component defines the security problem and finding solutions. It derives from a new paradigm in accident analysis called \Gls{stamp}.  This paradigm forms the basis for a hazard analysis methodology called \Gls{stpa}.  On top of \gls{stpa} is \Gls{stpasec}, which adds a security perspective to the hazard analysis. 

The second \gls{storm} component demonstrates trustworthiness.  \Gls{csbd} defines security as \gls{cia}.  It uses an access-control logic and computer-aided theorem proving to reason about, verify, and document security properties.

%\section{Certified Security by Design (CSBD)}\label{sec:intro:motivation}
%CSBD is an approach to verifying and documenting that a system's design satisfies the property of complete mediation. It uses formal methods by way of an access-control logic (\glsentryshort{acl}) that is implemented in an interactive theorem prover.  Formal methods add a mathematical rigor to the verification process.  The \glsentryshort{acl} is a formal propositional logic.  The interactive theorem prover takes the rigor a step further by using computer-aided reasoning in a trusted implementation.   The result is a set of formal proofs that both verify and document that the system's design is trustworthy with respect to complete mediation.  This set of formal proofs is verifiable and reproducible by independent third parties. 
%
%It is a straight forward process to apply CSBD.  First, security-sensitive objects are identified.  For example, a secured compound or bank account information are security-sensitive objects.  Then properties of access control are described in terms of authentication and authorization.  For example, authentication may reIy on ID badges or atm cards.   Authorization may be described in a policy that determines who is authorized to access the compound or who is authorized to access bank account funds.  Additionally, modification and access rights may also be described in terms of ordered security and integrity labels.  For example, security labels for a bank account may include r (read only) $\leq$ d (deposit) $\leq$  w (withdraw).  Account owners may be granted the highest integrity level of withdraw, whereas bank managers may only be granted read  access.   The owner can read, deposit, and withdraw funds from the account.  The bank manager can only read funds from the account.  Integrity labels may include st (standard service) $\leq$ gl (gold service) $\leq$ pr (premium service).  If standard services include checking account fees and gold services do not and the account owner has a gold services account, then the owner should not be subject to the standard services checking account fees.  Furthermore, standard services account holders should not be granted gold service privileges because it would lessen the value of those service.
%
% Next, the system is described in \glsentryshort{hol} wherein properties of complete mediation are verified.  This verification is a set of formal proofs that are included with the system's documentation.  Full disclosure of methods are included so that the proofs can be verified and reproduced by independent third parties.
%



\section{Extending The Range of Applicability}\label{sec:intro:motivation}

%CSBD is a core requirement for the Master of Science degree in Cybersecurity at Syracuse University.  It is also part of the Air Force Research Laboratory's STPA-Sec doctrine.   But, up until now, CSBD has been applied to automated systems.  For example, CSBD has been used with JPMC's SWIFT protocols \cite{pkm} and the F-18 Viper UAV payload controller and secure memory loader verifier (SMLV).  But, s

Systems engineering spans the range of fully automated to non-automated, human-centered systems.  Until now, \gls{storm} has been applied to automated systems.  This master thesis extends the range of \gls{storm} applicability by demonstrating it on a non-automated, human-centered system.  As an example of such a system, patrol base operations are chosen.  These operations are described in the U.S. Army Ranger Handbook \cite{rangermanual}.   

\section{This Master Thesis}\label{sec:thismasterthesis}
The chapter \ref{chp:storm} describes \gls{storm} and STPA/STPA-Sec.  The next chapter applies the STPA/STPA-Sec analysis on the patrol base operations.  Chapter \ref{chp:csbdacl} discusses CSBD, the access-control logic (ACL) and its implementation in the Higher Order Logic (HOL) Theorem Prover.


  The chapters begin with abstract descriptions of the patrol base operations in the context of the specific chapter.  Each chapter adds sufficient detail that subsequent chapters should be more easily understood if read in order.  Nonetheless, there are sufficient references to previous chapters if the reader wishes to skip around.


Chapter \ref{chp:background} discusses background material that may be helpful to the reader in understanding the following chapters.  Chapter \ref{chp:sse} discusses the hierarchical model of the patrol base operations in the context of systems engineering and systems security engineering.  Chapter \ref{chp:csbdacl} describes Certified Security by Design and the access-control logic in detail.  It also discusses the \glsentryshort{hol} implementation of the \glsentryshort{acl}.  Chapter \ref{chp:pb} describes the hierarchical model of the patrol base operations. Chapter \ref{chp:ssmmodel} describes secure state machines and their \glsentryshort{hol} implementation.  Chapter \ref{chp:pbssm} describes the implementation of several patrol base operations secure state machines in \glsentryshort{hol}.  Chapter \ref{chp:other} describes additional models that are discussed throughout the project, but not implemented in \glsentryshort{hol}.  Chpater \ref{chp:discussion} provides some discussion of the findings that are not covered in previous chapters.  Finally, Chapter \ref{chp:future} discusses future work and implications. 

The appendices include pretty-printed \glsentryshort{hol}-generated output of all the \glsentryshort{acl} theorems.  They also contain the code for all the \glsentryshort{hol} theorems both as pretty-printed \glsentryshort{hol}-generated output and the actual code.  There is also an appendix describing the folder structure of the files included with this thesis.


As the example at the start of this chapter makes clear, access control is a serious matter.  Systems should be designed with security from the start.   Certified Security by Design (\glsentryshort{csbd})  is an effective approach to designing trustworthy systems with respect to access control.  It does this using formal methods and computer-aided reasoning to verify and document the property of complete mediation.

The next chapter provides some background material that is relevant to subsequent chapters.

\end{document}