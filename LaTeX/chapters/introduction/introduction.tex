%introduction
\documentclass[../../main/main.tex]{subfiles}

\pagenumbering{arabic}

\begin{document}
\glsresetall

\chapter{Introduction}
 Imagine this scene from a movie titled "The Kingdom."  Its a bright, sunny weekend day in Saudi Arabia.  The compound is isolated from the rest of the Kingdom and heavily guarded by Saudi police.  The American civilian contractors are free to enjoy the American culture they are accustomed to without offending the locals.  A large group of contractors and their families are enjoying their day off watching a game of little league on the compound.  
 
 
Unexpectedly, two men dressed in Saudi police uniforms begin showering the crowd with machine gun fire.   Civilian men, women, and children run screaming for cover. The two gun men are eventually neutralized by Saudi police officers.  Saudi police yell to the survivors "come this way" to lure them away from the carnage. One of these men dressed in a police uniform shouts "Allah Akbar" and detonates, killing himself and several others near him. 

What went wrong?  How did these terrorists gain access to a secured compound?  

Movies like this and the news are bringing the reality of security to the forefront of everyone's mind.   Everyone is concerned with controlling access to their information, their property, and more importantly their lives.   The work discussed in this master thesis focuses on the critical property of access control in systems security engineering.

This chapter provides an introduction to this master thesis.  It briefly touches upon the field of systems security engineering and the subspecialty of access-control by way of complete mediation.  It then introduces an approach to verifying and documenting this property using the Certified Security by Design (\glsentryshort{csbd}) approach.  After introducing the \glsentryshort{csbd}, this chapter presents an application of \glsentryshort{csbd} that aims to fill a gap in its field of applicability.  This chapter goes on to discuss how \glsentryshort{csbd} is applied to a non-automated, human-centered military system (patrol base operations).  Finally, this chapter offers the reader a glimpse at what to expect in the subsequent chapters.



      %%%%%%%%%%%%%%%%%%%% Section Motivation %%%%%%%%%%%%%%%%%%%%%%
\section{In Context}\label{sec:intro:motivation}
The authority on standards in systems engineering is the International Organization for Standardization (ISO).  In one of its published standards (ISO/IEC/IEEE 25288) it describes systems engineering guidelines for managing man-made systems.  These man-made systems span the range from human-centered systems to fully automated systems.  An example of a human-centered system is a partially- or non-automated system for controlling access to a base or compound.  An example of a fully automated system is the computer that controls a driverless car.  In all these systems, security is often a critical component of the systems engineering process.


The recognized authority in systems security engineering, a subspecialty of systems engineering, is the National Institute for Standards and Technology (NIST) Special Publication 800-160 (vol 1 and vol 2).  It describes a framework for designing trustworthy systems by engineering security into the system from the start.  A subspecialty of systems security engineering manages access to security-sensitive objects using the principle of complete mediation.  Complete mediation broadly refers to verifying and documenting authentication and authorization of access to security sensitive objects. 


\subsection{Systems Are Everywhere}

\section{This Master Thesis}\label{sec:thismasterthesis}
\glsresetall
This master thesis describes a method for designing secure systems.  The method is called \glsfirst{csbd}. \glsunset{csbd}  \glsname{csbd} has been successfully demonstrated on automated systems such as ... and ....  But, until this research, it has not been demonstrated on non-automated, human-centered systems.  

Systems span the range of fully automated to fully non-automated.  This master thesis focuses on one end of this range: non-automated, human-centered systems.


\begin{itemize}
\item The first question addressed in this master thesis is whether \gls{csbd} could be successfully applied to non-automated, human-centered systems.  This is the primary objective.  An example of a non-automated, human-centered system is the patrol base operations defined in the United States Army Ranger Handbook\cite{rangermanual}.   Patrol base operations exemplify a non-automated, human-centered system wherein security is critical to mission success.  In this master thesis, the results of applying CSBD to patrol base operations is discussed. 

\item The patrol base operations are also an example of a predefined system.  This means that this thesis also addresses the question of whether CSBD could be successfully applied to a pre-designed, non-automated, human-centered system.  This is important because many such systems in use today are already designed and implemented.  CSBD demonstrates that it can verify and document the security properties of current, in-use systems.  

\item These thesis describes a hierarchy of secure state machines (SSMs) used to model the patrol base operations.  This approach demonstrates that formal methods can be applied to large scale and complicated systems.  The hierarchy manages patrol base operations by successful levels of decreasing abstraction.  Each level in the hierarchy consists of one or more SSMs.  Each SSM is modularized and models one aspect of the patrol base operations at one level of abstraction.  The levels and modules are connected together by an OMNI level, all-seeing, principal.  Each module only needs to be aware of this OMNI level principal.  They do not need to be aware of the details of any other module.  With this divide-and-conquer approach, CSBD can be readily applied to large and complicated systems.

\item The successful application of CSBD to patrol base operations also suggests its use in combining automation with human-centered systems.  The approach employed by this master thesis involves describing the patrol base operations as a hierarchy of secure state machines.  This hierarchy has the property that it is easy to demonstrate security properties of the system, which is the goal of CSBD.  But, it also has the property that it describes the patrol base operations as a system that is amiable to automation.  Such automations of pre-defined non-automated, human-centered systems could include, for example, accountability systems for tracking supplies and personal.  

In the not-so-distant future, the military, in particular, will most likely seek tracking and accountability systems for pre-existing, non-automated military operations.  These systems, like all security-sensitive military systems, should be designed according to NIST 800-160 standards.  These standards require the formal verification and documentation provided by CSBD and demonstrated in this thesis.   

\end{itemize}


\end{document}