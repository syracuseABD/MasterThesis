%sse
\documentclass[../../main/main.tex]{subfiles}


\begin{document}
\title{Systems Security Engineering}


%%%%%%%%%%%%%%%%%%%%% Chapter CSBD ACL %%%%%%%%%%%%%%%
\chapter{Systems Security Engineering \& Patrol Base Operations}\label{chp:sse}


       %%%%%%%%%%%%%%%%%% Section Systems %%%%%%%%%%%%%%%%
\section{The Systems Perspective}\label{sec:systems}
A system is a set of interacting and interdependent components that act as a whole to perform some behavior or function.  Examples of systems include the human body, socio-political systems, and computer systems. 

The patrol base operations satisfy this definition of a system.  As a whole, the patrol base operations perform some function(s).  This function is described in the Ranger Handbook \cite{rangermanual} and discussed in section \ref{sec:pb}.  The patrol base operations are comprised of interdependent and interacting components.  In general these components are the individual soldiers.  But, the way this master thesis defines the patrol base operations, the definition of a component varies.

This master thesis defines the patrol base operations as a system of systems.  More specifically, this thesis models the patrol base operations as a hierarchy of \glsentrylongpl{ssm}.  Chapter \ref{chp:ssmmodel} describes \glsentryshortpl{ssm} in general.   Section \ref{sec:modelingpb} describes this model of the patrol base operations.  This model presents the patrol base operations as a hierarchy wherein each level of the hierarchy represents a decreasing level of abstraction.   

At the top and most abstract level, the components are phases of the patrol base operations.  These phases commence in a sequential order to achieve the goal of the patrol base operations.  Each lower level of the hierarchy is composed of less abstract phases. At each level, the components function sequentially (typically) to achieve the ultimate goal.  

This system of system also contains non-hierarchically defined components.  For example, an escape-level component models situations wherein the patrol base operations are aborted.  The escape level component is reachable from any component at any level of the hierarchy.  Soldiers also function within this system of systems in a non-hierarchical manner.  However, soldiers were not modeled in detail for this master thesis.  Nevertheless, they were discussed in detail and ready to be modeled.

In this way, the patrol base operations represent a system and are amiable to the systems engineering perspective.  

     %%%%%%%%%%%%%%%%%% Section Systems Engineering %%%%%%%%%%
\section{Systems Engineering}\label{sec:se}
Systems engineering is an interdisciplinary approach aimed at solving problems involved in the design, development, realization, and life-cycle maintenance of systems.
  
  
 This master thesis focuses on the design phase of systems engineering.   The aim is to model the patrol base operations in a manner that is amiable to verifying specific security properties of the system.  Specifically, the patrol base operations must satisfy the property of complete mediation. 


But, this thesis does not aim to build a new system.  Rather this thesis remodels an existing system.   This is necessary because the goal of this thesis is to determine whether or not it is possible to verify the specific security properties on the subclass of systems that we are exploring.  Most people would agree that testing a new method on a new system would be unwise.  This is why this thesis did not do that.    


This approach has the side-effect of also demonstrating CSBDs utility in the life-cycle phase of systems engineering.  It follows readily from the news today that many systems were not designed with security in mind.  Eliminating already-in-use systems and legacy systems may not always be practical or desirable.  Nevertheless, security remains an important aspect of system performance.  This, in part, justifies a re-look at (or remodeling of)  of an already existing system.


There are additional benefits to systematically modeling the patrol base operations in a way that is amiable to formal verification.  This type of thinking provides new insights and suggests areas for improvement\footnote{The subject matter expert who focused on the details of the patrol base operations also noted areas for improvement.  He was not available to provide details at the writing of this master thesis.}.  This is a known benefit.  For example, Wikipedia \cite{wikiformalmethods} notes that "Sometimes, the motivation for proving the correctness of a system is not the obvious need for reassurance of the correctness of the system, but a desire to understand the system better."  A greater understanding of a system applies to all phases of systems engineering.

     %%%%%%%%%%%%%%%%%% Section Systems Security Engineering %%%%%%
\section{Systems Security Engineering}\label{sec:sse} (Primary source for this section is \cite{NIST800160})

Systems security engineering (\glsentryshort{sse}) is a sub-discipline of systems engineering.  Figure \ref{fig:nist800160} shows \glsentryshort{sse} in relation to systems engineering and other sub-disciplines of \glsentryshort{sse}.  This master thesis falls into one of the Security Specialties in this diagram.

\begin{figure}[h]
\includegraphics[width=\linewidth]{../figures/seincontext.png}
\caption{\label{fig:nist800160}Systems security engineering in relation to systems engineering. (Image from \glsentryshort{nist} Special Publication 800-160: Systems Security Engineering Considerations for a Multidisciplinary Approach in the Engineering of Trustworthy Secure Systems.)}
\end{figure}

According to \glsentryshort{nist} Special Publication 800-160, "Systems security engineering focuses on the protection of stakeholder and system assets so as to exercise control over asset loss and the associated consequences."  Three key concepts in \glsentryshort{sse} are stakeholder, asset, and unacceptable losses.  In modeling the patrol base operations, this thesis first defines these key concepts.  

\begin{description}
\item[ Stakeholder]  The stakeholder controls the design of the system.  The stakeholder defines what the system should do.  The stakeholder also defines what the system should not do and what are unacceptable losses.  The stakeholder for the patrol base operations are ultimately the U.S. military.  This was critical to the original design of the patrol base operations.  But, this thesis has a different purpose, that of demonstrating specific security properties of the patrol base operations using CSBD.  These security properties are that of complete mediation.  For this master thesis, the stakeholders are everyone involved in this research.  
\item[Asset] An asset is anything that is of value to the stakeholder.   In the patrol base operations, this includes soldiers, equipment, and the mission.
\item[Unacceptable losses] Unacceptable losses are self-defining.  Unacceptable losses for the patrol base operations are defined broadly as any event that would cause the patrol base operation as a whole to abort.  These are: contact with the enemy, casualties, a change in mission from higher-up.
\end{description}

It is also a critical objective of \glsentryshort{sse} to identify and define the security goals of the stakeholder in a way that minimizes asset loss and avoids unacceptable losses.  The security properties of the patrol base operations are already built-in to the design of the operations from the Ranger Handbook.  These are undoubtedly the result of years of military expertise.  Our goal is not to define the security features of the patrol base operations, but to describe them in manner amiable to verification of complete mediation.  Identification of assets and unacceptable losses from the Ranger Handbook is sufficient to do this.

To cover the unacceptable losses, this master thesis models an escape-level \glsentrylong{ssm} (SSM).  If at any phase in the patrol base operations any authenticated principal (i.e., the platoon leader) reports an abortable event, the escape-level \glsentryshort{ssm} will abort the patrol base operations. This includes casualties or unacceptable equipment failure.  By creating one escape-level \glsentryshort{ssm}, this thesis creates an modularized yet expandable treatement of unacceptable losses.


To further describe the model of the patrol base operations in the context of \glsentryshort{sse}, a few more concepts are necessary.  

\subsubsection{Systems Security Engineering Framework}\label{sssec:sseframework}
NIST 800-160 describes the systems security engineering framework shown in figure \ref{sseframework} as a "contexts within which systems security engineering activities are conducted."  CSBD focuses primarily on demonstrating trustworthiness.  Nevertheless, this master thesis also addressed other aspects of the framework.  

The problem and solution phases for the patrol base operations take a different approach in the context of this master thesis. It is not our goal to outline the potential security threats and then to find solutions for them.  This was, hopefully, already done when the patrol base operations were originally defined.  Nonetheless, it is necessary to identify the problem and solution within the patrol base operations in order to verify their security properties.  

The security objective from the perspective of this master thesis is to ensure complete mediation of all actions.  Complete mediation requires that we identify objects to be accessed, the accessors, the means of authenticating the accessors, and the a policy for authentication.  

The objects to be accessed varies at each level of abstraction. At the top and most abstract level, the objects are the transitions.  That is, control over the transitions between phases (or states) of the patrol base operations are to be protected.  

The accessors are the leaders designated at each level of abstraction.  The best way to negotiate access to objects is by the principle of least privilege.  This principle assigns privilege to only those who are deemed necessary and to no others.  At the top level of the patrol base operations model, the platoon leader (and only platoon leader) is designated as the accessor. This means that only the platoon leader can issue commands to move-on to the next phase (or state) in the patrol base operations. 

In the real patrol base operations, soldiers recognize each other by face.  Thus, it is sufficient to just authenticate the platoon leader at each level without requiring any passwords or other means of authentication\footnote{The author and subject matter expert discussed authentication in the context of an accountability system wherein passwords or chips were required for authentication.  But, we did not implement such authentication techniques in this project.}.

Policy is determined for each module and at each level of the patrol base operations model.  The model includes not only "who" is authorized to do what, but also what preconditions are necessary prior to authorization.  At the top level, there are only two requirements for the platoon to move to the next phase of the operations: (1) the previous phase was complete, and (2) the platoon leader issue the command to move to the next phase.  This creates a potential conflict between the goal of modularizing each component and creating a hierarchy wherein each level of the hierarchy expands upon the one above it.  

\begin{figure}[h]
\includegraphics[width=\linewidth]{../figures/sseframework}
\caption{\label{sseframework}Systems security engineering Framework. (Image from \glsentryshort{nist} Special Publication 800-160: Systems Security Engineering Considerations for a Multidisciplinary Approach in the Engineering of Trustworthy Secure Systems.)}
\end{figure}

\subsubsection{Verifying \& Documenting Trustworthiness}\label{sssec:sseframework}

\paragraph*{Trustworthiness}

\paragraph*{Verification}

\paragraph*{Documentation}
A common idiom of science and engineering is "if you didn't document it then it didn't happen."  There is some leeway in engineering because the presence of a system is indicative that "you did it." However, this hides the important details regarding the security of the system.  Maybe "you did it" but "did you do it right?"  This is where documentation is important. 

'''systems security engineering provides a sufficient base of evidence that supports claims that the desired level of trustworthiness has been achieved..."


\subsection{Trustworthiness}\label{ssec:trustworthiness}
\subsubsection{Complete Mediation}\label{sssec:completemediation}

\subsection{Verification}\label{ssec:verification}
\subsection{Documentation}\label{ssec:documentation}
\subsection{Reproducibility}\label{ssec:reproducibility}

      %%%%%%%%%%%%%%%%%%% Section Verification & Documentation %%%%%%
\section{Verification \& Documentation}

      %%%%%%%%%%%%%%%%%%% Section Complete Mediation %%%%%%%%%%%
\section{Principle of Complete Mediation}

\subsection{Formal Verification Using Computer-Aided Reasoning}

\end{document}