%ssms

\documentclass[../../main/main.tex]{subfiles}




\begin{document}
\title{Secure State Machine Model}

%%%%%%%%%%%%%%%%%%%%% Chapter SSM Model %%%%%%%%%%%%%%%
\chapter{Secure State Machine Model}\label{chp:ssmmodel}

      %%%%%%%%%%%%%%%%%%% Section State Machines %%%%%%%%%%%%%
\section{State Machines}\label{sec:sm}
\subsection{Transition Commands}
\subsection{Next-state Function}
\subsection{Next-output Function}
\subsection{Configuration}

      %%%%%%%%%%%%%%%%%%% Section Secure State Machines %%%%%%%%%
\section{Secure State Machines}\label{sec:ssm}
Secure state machines add a level of security to the state machines model.  In particular, the secure state machine implements access control by way of complete mediation.   

\subsection{State Machine Versus Secure State Machine}
State machines define states, inputs, outputs, next-state functions, and next-output functions.  Through these means, the state machine defines the behavior of the state machine.  Secure state machines, on the other hand, add the concepts of complete mediation to the state machine model by including checks on authentication and authorization.

 
\subsection{Monitors}\label{monitors}
It is the duty of the monitor to control the behavior of the \glsentryshort{ssm} with regards to complete mediation.  The monitor is a essentially a guard that checks for the proper authentication and authorization for all \glsentryshort{ssm} transition requests.  In a real world analogy, the monitor is the sentry at the gate who checks IDs and determines who is granted access to the base.  

\subsection{Transition Types}
The monitor assigns a transition type to each command (or \glsentryshort{ssm} transition request). This assignment is based on the security policy and the rules for authentication.  There are three transition types: execute (\textit{exec}), trap, (\textit{trap}), and discard (\textit{discard})\footnote{The names are derived from their use in virtual machines.  Commands in virtual machines are either executed, trapped, or discarded.  Each has a different behavior in the machine.}.  

\subsubsection{\textit{exec}}
The \textit{exec} transition type indicates that a command should be executed.  For example, if the Platoon Leader issues the command (request) \textit{crossLD}, then the monitor must first check the authentication and authorization for that request.  If the monitor authenticates the Platoon Leader and authorizes her on that request, then the monitor can justify executing that request.  In the \glsentryshort{ssm} model, this means that the transition from the PLAN_PB state to the MOVE_TO_ORP state (indicated by the command \textit{crossLD}) should be allowed. 
\subsubsection{\textit{trap}}
The \textit{trap} transition type indicates that a command should NOT be executed.  The \textit{trap} transition type indicates that the principal is authenticated, but NOT authorized on that command.  For example, if the Platoon Leader issues the command \textit{initiateMovement}, then the monitor must first check the authentication and authorization of that request.  In this case, the monitor authenticates the Platoon Leader.  But, the monitor does not authorize the Platoon Leader on the command \textit{initiateMovement} because only the Platoon Sergeant is authorized on this command.  In the \glsentryshort{ssm} model, this means that the transition from the WARNO state to the REPORT1 state (partially indicated by t\textit{initiateMovement}) should NOT be allowed.  It should be trapped.


\subsubsection{\textit{discard}}
Like the \textit{trap} transition type, the \textit{discard} transition type indicates that the command should not be executed.  The \textit{discard} transition type indicates that the principal is neither authenticated nor authorized.  It may also indicate that the command is not of the correct form.  In either case, transition types are not executed.  For example, if SomeGuy issues \textit{anyCommand}, the monitor must first check the authentication and authorization for that request. If the monitor does not authenticate SomeGuy, then SomeGuy's command is discarded.

It is useful to differentiated between the \textit{trap} and \textit{discard} transition types.  This allows for the behavior of the \glsentryshort{ssm} to handle authenticated but unauthorized and un-authenticated and unauthorized commands differently.  For example, a sentry may choose to deny access to someone who is authenticated but unauthorized.  On the other hand, a sentry may choose to detain an unauthenticated person attempting to gain access to a base.

\subsection{Commands}
Commands in the \glsentryshort{ssm} are handled differently than in the state machine.  Principals issue commands (make requests).  The monitor inspects the command (request) for proper authentication and authorization and assigns a transition type to the command (request).  This combination of transition type and command (request) is then passed to the next-state and next-output functions.  These functions define how the \glsentryshort{ssm} responds to each transition type and command (request) pair.  This differs from the state machine in that the state machine only defines next-state and next-output functions for commands.  The state machines is not concerned with authentication and authorization (i.e., access-control).

For example, the \glsentryshort{ssm} for the top level is shown in figure \ref{ssmPBDiagram2} (this is the same as figure \ref{ssmPBDiagram} in section \ref{ssec:toplevel}).  

\begin{figure}[h!]
\centering
\includegraphics[width=\textwidth]{../figures/ssmPBDiagram}
\caption{\label{ssmPBDiagram2} Top level diagram.}
\end{figure}

In the state machine, the transition from PLAN_PB to MOVE_TO_ORP only requires the command \textit{crossLD}.  But, in the secure state machine, transitions require some form of access-control.  Transitions are indicated by principals making requests of the form \textit{SomePrincipal says someCommand}.  In figure \ref{ssmPBDiagram2}, the Platoon Leader makes a request (issues a command) to transition from the PLAN_PB state to the MOVE_TO_ORP by stating 
 \[\textit{PlatoonLeader says crossLD}.\]

The monitor then checks the authentication and authorization of the principal and returns the command with a transition type.  For example, if the Platoon Leader is both authenticated and authorized on the command \textit{crossLD}, then the monitor returns the transition type and command pair 
\[exec \hspace{0.2cm} crossLD\]

This is passed to the next-state and next-output functions.  The next-state function is then justified in executing the transition from the PLAN_PB state to the MOVE_TO_ORP state.  

\subsection{Authentication}
Authentication in the \glsentryshort{ssm} refers to verification of identity.  Authentication is a very broad topic and details are beyond the scope of this master thesis.  

In this master thesis, authentication is dealt with by simple visual confirmation of a principal's identity\footnote{Although, other methods are discussed in the chapter \ref{chp:other}, particularly with regards to accountability systems.}.  This was justified by assuming that most soldiers in the platoon should recognize their leaders.

Authentication is verified by the monitor on each request\footnote{In the implementation of the \glsentryshort{ssm}, this may or may not be the case.  For example, access to a secured facility may require proper authentication upon entering the facility.  This may be indicated with by a badge worn by the individual while in the facility.  Within the facility, only badges are checked.  But, the process of verifying identity only occurs at the entry.}.  This means that each request must be of the form \textit{SomePrincipal says someCommand}, even if the principal was previously authenticated.  

\subsection{Authorization}
Authentication is a way of controlling who has access to what by means of a security policy.  The security policy implements the principal of complete mediation (and other security policies).  

Authentication in the \glsentryshort{ssm} is typically one or more functions that define which principals have control over which transitions.  A simple authorization in a \glsentryshort{ssm} is \textit{SomePrincipal controls someCommand}.    Using the access-control logic (ACL) described in Chapter \ref{chp:csbdacl} and a request of the form \textit{SomePrincipal says someCommand}, \textit{someCommand} is justified.  Of course, the monitor must first check the authentication of SomePrincipal.  Once that is verified, the monitor returns the transition type and command pair \textit{exec someCommand}.

Authorization may be more complicated.  For example, transition from the PLAN_PB state to the MOVE_TO_ORP state in 

\subsection{Next-state And Next-output Functions}
The next-state and next-output functions define the behavior for the \glsentryshort{ssm}.  Whereas the behavior of the state machine is defined only for states and commands, the behavior in the \glsentryshort{ssm} includes the behavior for each transition type in combination with each command.  This means that for each command, the next-state and next-output functions define three separate behaviors: \textit{exec someCommand}, \textit{trap someCommand}, and \textit{discard someCommand}.  

\subsection{Configurations: five parts}
Secure state machine configurations consist of five parts.


      %%%%%%%%%%%%%%%%%%% Section SSMs in HOL %%%%%%%%%%%%%
\section{Secure State Machines in HOL}\label{sec:sminHOL}

\subsection{Parameterizable Secure State Machine}
\subsection{Input Stream}
The input to the secure state machines is in the form of a list of inputs.  Typically, these inputs are of the form \textit{P says prop (SOME cmd)}.  It is necessary to extract particular components from the list and list elements.  Several functions are defined to do this.  They are described below.  These are essentially helper functions.

\paragraph*{extractCommand}
extractCommand takes one input of the form \textit{P says prop (SOME cmd)} and extracts the \textit{cmd} part.
\begin{tabbing}
\parskip=8pt
\HOLTokenTurnstile{} \\
\hspace{0.3cm}\HOLConst{extractCommand} (\HOLFreeVar{P} \HOLConst{says} \HOLConst{prop} (\HOLConst{SOME} \HOLFreeVar{cmd})) \HOLSymConst{=} \HOLFreeVar{cmd}
\parskip=18pt
\end{tabbing}

\paragraph*{commandList}
commandList takes an input list consisting of list elements of the form \textit{P says prop (SOME cmd)}.  It returns a list of all the \textit{cmd} elements.
\begin{tabbing}
\parskip=8pt
\HOLTokenTurnstile{} \HOLSymConst{\HOLTokenForall{}}\HOLBoundVar{x}. \\
\hspace{0.3cm} \HOLConst{commandList} \HOLBoundVar{x} \HOLSymConst{=} \HOLConst{MAP} \HOLConst{extractCommand} \HOLBoundVar{x}
\parskip=18pt
\end{tabbing}

\paragraph*{extractPropCommand}
extractPropCommand takes one input of the form \textit{P says prop (SOME cmd)} and extracts the \textit{prop (SOME cmd)} part.
\begin{tabbing}
\parskip=8pt
\HOLTokenTurnstile{}\\
\hspace{0.3cm} \HOLConst{extractPropCommand} (\HOLFreeVar{P} \HOLConst{says} \HOLConst{prop} (\HOLConst{SOME} \HOLFreeVar{cmd})) \HOLSymConst{=} \HOLConst{prop} (\HOLConst{SOME} \HOLFreeVar{cmd})
\parskip=18pt
\end{tabbing}

\paragraph*{propCommand}
propCommand takes an input list consisting of list elements of the form \textit{P says prop (SOME cmd)}.  It returns a list of all the \textit{prop (SOME cmd)} elements.
\begin{tabbing}
\parskip=8pt
\HOLTokenTurnstile{} \HOLSymConst{\HOLTokenForall{}}\HOLBoundVar{x}. \\
\hspace{0.3cm} \HOLConst{propCommandList} \HOLBoundVar{x} \HOLSymConst{=} \HOLConst{MAP} \HOLConst{extractPropCommand} \HOLBoundVar{x}
\parskip=18pt
\end{tabbing}


\paragraph*{extractInput}
extractInput takes one input of the form \textit{P says prop x} and extracts the \textit{x} part. Note that \textit{x} can have two forms: \textit{NONE} or \textit{SOME cmd}.
\begin{tabbing}
\parskip=8pt
\HOLTokenTurnstile{}\\
\hspace{0.3cm} \HOLConst{extractInput} (\HOLFreeVar{P} \HOLConst{says} \HOLConst{prop} \HOLFreeVar{x}) \HOLSymConst{=} \HOLFreeVar{x}
\parskip=18pt
\end{tabbing}

\paragraph*{inputList}
inputList takes an input list consisting of list elements of the form \textit{P says prop x}.  It returns a list of all the \textit{x} elements.
\begin{tabbing}
\parskip=8pt
\HOLTokenTurnstile{} \HOLSymConst{\HOLTokenForall{}}\HOLBoundVar{xs}. \\
\hspace{0.3cm} \HOLConst{inputList} \HOLBoundVar{xs} \HOLSymConst{=} \HOLConst{MAP} \HOLConst{extractInput} \HOLBoundVar{xs}
\parskip=18pt
\end{tabbing}


\subsection{Commands}
\paragraph*{Option Type}
The option type allows for the return of "no value."  In functional programming, this is an important concept because functions must always return a value.  Consider the following example.
Consider a search for the letter associated with the number 1 in an association list of the form \{('a',1), ('b',2), ... (a,26)\}. In this case, the search returns the integer value 'a'.  But, what if the target of the search is the number 27?  The search must return something.  But, there is nothing to return.  This is where the option type comes in handy.  Instead of returning 'a' for 1 and 'b' for 2 and so on, the search returns SOME 'a' and SOME 'b' and so on.  If the search does not find a match, it returns NONE.

The definition for the option datatype is shown below.

\HOLFreeVar{option} = \HOLConst{NONE} \HOLTokenBar{} \HOLConst{SOME} 'a

In the datatype definition above, 'a is replaced with some other datatype.  

\paragraph*{A Closer Look at Commands}
To see how this works, it is necessary to define some other datatypes. The OMNILevel folder in OMNITypesScript.sml contains definitions that are used in all secure state machines. One definition is the \HOLFreeVar{command} datatype definition.

\begin{tabbing}
\HOLFreeVar{command} = \HOLConst{ESCc} \HOLTyOp{escCommand} \HOLTokenBar{} \HOLConst{SLc} 'slCommand
\end{tabbing}

The \HOLFreeVar{command} datatype consists of two additional datatypes: \HOLConst{ESCc} \HOLTyOp{escCommand} and \HOLConst{SLc} 'slCommand.  Note that the first part of each of these are the datatype constructors\footnote{see the background section \ref{sec:adtinml}}: \HOLConst{ESCc} and \HOLConst{SLc}.  The second part is the name of the datatype variable\footnote{Both of these are datatype variables because they define other datatypes.} or datatype.  

The first datatype refers to the escape commands.  They are defined as \HOLTyOp{escCommand} in the same file as \HOLFreeVar{command}.

 \begin{tabbing}
 \HOLFreeVar{escCommand} = \= \HOLConst{returnToBase} \\
 						\>\HOLTokenBar{} \HOLConst{changeMission} \\
						\>\HOLTokenBar{} \HOLConst{resupply}
           \HOLTokenBar{} \HOLConst{reactToContact}
\end{tabbing}

This datatype definition defines three commands (or datatype values) which represent escape conditions in the patrol base operations.  

The second dataytpe variable 'slCommand refers to the state-level commands.  These are defined further in each secure state machine.  

Notice that there is a tick mark (apostrophe) before 'slCommand and not before \HOLTyOp{escCommand}.  In general, the tick mark in \glsentryshort{hol} represents an undefined dataytpe.  In this case, 'slCommand is not yet defined (because it is defined elsewhere), whereas the definition for \HOLTyOp{escCommand} is defined in the same file and above the definition for \HOLFreeVar{command}.   

An example of a definition for 'slCommand can be found in the top level \glsentryshort{ssm}.  It is defined in the folder topLevel and in the file PBIntegratedTypeScript.sml file.

\HOLFreeVar{slCommand} = \HOLConst{PL} \HOLTyOp{plCommand} \HOLTokenBar{} \HOLConst{OMNI} \HOLTyOp{omniCommand}

This is defined similarly to \HOLFreeVar{command}.  There are two datatypes that make-up the datatype \HOLFreeVar{slCommand}.  None of these have tick marks, which means both of these are defined.  In particular, they are both defined in the same file as \HOLFreeVar{slCommand}.

\HOLFreeVar{plCommand} refers to the Platoon Leader commands.  These are commands that the Platoon Leader is authorized to make.  

\begin{tabbing}
\HOLFreeVar{plCommand} = \= \HOLConst{crossLD} \\
					     \>\HOLTokenBar{} \HOLConst{conductORP} \\
					     \>\HOLTokenBar{} \HOLConst{moveToPB} \\
					     \>\HOLTokenBar{} \HOLConst{conductPB}\\
         				     \> \HOLTokenBar{} \HOLConst{completePB} \\
				             \> \HOLTokenBar{} \HOLConst{incomplete}
\end{tabbing}


\HOLFreeVar{omniCommand} refers to commands that the OMNI level principal\footnote{See section \ref{ssec:omnilevel} for a discussion of the OMNI level principal.} is authorized to make. 

\begin{tabbing}
\HOLFreeVar{omniCommand} = \= \HOLConst{ssmPlanPBComplete} \\
						 \> \HOLTokenBar{} \HOLConst{ssmMoveToORPComplete}\\
 				       	 	 \> \HOLTokenBar{} \HOLConst{ssmConductORPComplete} \\
						 \> \HOLTokenBar{} \HOLConst{ssmMoveToPBComplete}\\
            					 \> \HOLTokenBar{} \HOLConst{ssmConductPBComplete}\\
					 	 \> \HOLTokenBar{} \HOLConst{invalidOmniCommand}
\end{tabbing}          
          
          
\paragraph*{Option Type with Commands}
With these definitions, it is possible to see how the options types are used with commands (datatypes).  What follows is a list of examples using the option types and commands (datatypes) described above.  The type signatures are also included because it will help the reader recognize them in the \glsentryshort{hol} code.
\begin{description}
\item[ ] SOME returnToBase \\
The type for this is (escCommand command)Option.
\item[ ] SOME moveToORP  \\
The type for this is ((plCommand slCommand) command)Option.  
\item[ ] SOME ssmMoveToORPComplete\\
The type for this is ((omniCommand slCommand) command)Option.  
\end{description}

Note that in the \glsentryshort{hol} code for the patrol base operations, the reader will typically see (slCommand command)Option.  This is because the definitions require a type \HOLFreeVar{slCommand}, which includes \HOLFreeVar{plCommand}  and \HOLFreeVar{omniCommand}.




\subsubsection{Transition Types}
Transition datatypes indicate how a command is handled by the monitor.  The three transition datatypes are described below. 
\HOLFreeVar{trType} = \HOLConst{discard} 'cmdlist \HOLTokenBar{} \HOLConst{trap} 'cmdlist \HOLTokenBar{} \HOLConst{exec} 'cmdlist




\subsubsection{Command Type}
\subsection{Authentication}

\begin{tabbing}
\parskip=8pt

\HOLTokenTurnstile{} \=\HOLSymConst{\HOLTokenForall{}}\HOLBoundVar{elementTest} \HOLBoundVar{x}. \\
    \> \HOLConst{authenticationTest} \HOLBoundVar{elementTest} \HOLBoundVar{x} \\
    \> \HOLSymConst{\HOLTokenEquiv{}} \\
    \> \HOLConst{FOLDR} (\HOLTokenLambda{}\HOLBoundVar{p} \HOLBoundVar{q}. \HOLBoundVar{p} \HOLSymConst{\HOLTokenConj{}} \HOLBoundVar{q}) \HOLConst{T} (\HOLConst{MAP} \HOLBoundVar{elementTest} \HOLBoundVar{x})
\parskip=18pt
\end{tabbing}
     
\subsection{Authorization}
\subsection{Next-state Function}
\subsection{Next-output Function}

\subsection{Configurations}
The five-part representation of configurations in \glsentryshort{hol} are shown below.


\begin{tabbing}
\parskip=8pt
\HOLFreeVar{configuration} = \\
\hspace{0.3cm}    \HOLConst{CFG} \\
\hspace{0.5cm}  (('command \HOLTyOp{option}, 'principal, 'd, 'e) \HOLTyOp{Form} -> \HOLTyOp{bool})\\
\hspace{0.5cm}    ('state -> ('command \HOLTyOp{option}, 'principal, 'd, 'e) \HOLTyOp{Form} \HOLTyOp{list} -> \\
\hspace{0.5cm}    ('command \HOLTyOp{option}, 'principal, 'd, 'e) \HOLTyOp{Form} \HOLTyOp{list}) \\
 \hspace{0.5cm}   (('command \HOLTyOp{option}, 'principal, 'd, 'e) \HOLTyOp{Form} \HOLTyOp{list} -> \\
\hspace{0.5cm}    ('command \HOLTyOp{option}, 'principal, 'd, 'e) \HOLTyOp{Form} \HOLTyOp{list}) \\
 \hspace{0.5cm}  (('command \HOLTyOp{option}, 'principal, 'd, 'e) \HOLTyOp{Form} \HOLTyOp{list} \HOLTyOp{list}) \\
\hspace{0.5cm}   'state \\
\hspace{0.5cm} ('output \HOLTyOp{list})
\parskip=18pt
\end{tabbing}

\subsubsection{State Interpretation}
\subsubsection{Security context}
\subsubsection{Input stream}
\subsubsection{State}
\subsubsection{Output stream}
\subsection{Configuration Interpretation}

\begin{tabbing}
\parskip=8pt
\HOLTokenTurnstile{} \HOLConst{CFGInterpret}\\
\hspace{0.5cm}(\HOLFreeVar{M}\HOLSymConst{,}\HOLFreeVar{Oi}\HOLSymConst{,}\HOLFreeVar{Os})
     (\HOLConst{CFG} \HOLFreeVar{elementTest} \HOLFreeVar{stateInterp} \HOLFreeVar{context} (\HOLFreeVar{x}\HOLSymConst{::}\HOLFreeVar{ins}) \HOLFreeVar{state}
        \HOLFreeVar{outStream}) \HOLSymConst{\HOLTokenEquiv{}} \\
\hspace{0.5cm}(\HOLFreeVar{M}\HOLSymConst{,}\HOLFreeVar{Oi}\HOLSymConst{,}\HOLFreeVar{Os}) \HOLConst{satList} \HOLFreeVar{context} \HOLFreeVar{x} \HOLSymConst{\HOLTokenConj{}} (\HOLFreeVar{M}\HOLSymConst{,}\HOLFreeVar{Oi}\HOLSymConst{,}\HOLFreeVar{Os}) \HOLConst{satList} \HOLFreeVar{x} \HOLSymConst{\HOLTokenConj{}}\\
\hspace{0.5cm}(\HOLFreeVar{M}\HOLSymConst{,}\HOLFreeVar{Oi}\HOLSymConst{,}\HOLFreeVar{Os}) \HOLConst{satList} \HOLFreeVar{stateInterp} \HOLFreeVar{state} \HOLFreeVar{x}
\parskip=18pt
\end{tabbing}

\paragraph*{rule0}
\parskip=8pt
\begin{tabbing}
\HOLTokenTurnstile{} \\
\hspace{0.3cm}\HOLConst{TR} (\HOLFreeVar{M}\HOLSymConst{,}\HOLFreeVar{Oi}\HOLSymConst{,}\HOLFreeVar{Os}) (\HOLConst{exec} (\HOLConst{inputList} \HOLFreeVar{x}))\\
\hspace{0.5cm}(\HOLConst{CFG} \\
\hspace{0.9cm}\HOLFreeVar{elementTest} \\
\hspace{0.9cm}\HOLFreeVar{stateInterp} \\
\hspace{0.9cm}\HOLFreeVar{context} \\
\hspace{0.9cm}(\HOLFreeVar{x}\HOLSymConst{::}\HOLFreeVar{ins}) \\
\hspace{0.9cm}\HOLFreeVar{s} \\
\hspace{0.9cm}\HOLFreeVar{outs})\\
\hspace{0.5cm}(\HOLConst{CFG} \\
\hspace{0.9cm}\HOLFreeVar{elementTest} \\
\hspace{0.9cm}\HOLFreeVar{stateInterp} \\
\hspace{0.9cm}\HOLFreeVar{context} \\
\hspace{0.9cm}\HOLFreeVar{ins}\\
\hspace{0.9cm}(\HOLFreeVar{NS} \HOLFreeVar{s} (\HOLConst{exec} (\HOLConst{inputList} \HOLFreeVar{x})))\\
\hspace{0.9cm}(\HOLFreeVar{Out} \HOLFreeVar{s} (\HOLConst{exec} (\HOLConst{inputList} \HOLFreeVar{x}))\HOLSymConst{::}\HOLFreeVar{outs})) \\
\hspace{0.3cm}\HOLSymConst{\HOLTokenEquiv{}}\HOLConst{authenticationTest} \HOLFreeVar{elementTest} \HOLFreeVar{x} \\
\hspace{0.3cm}\HOLSymConst{\HOLTokenConj{}}\HOLConst{CFGInterpret} (\HOLFreeVar{M}\HOLSymConst{,}\HOLFreeVar{Oi}\HOLSymConst{,}\HOLFreeVar{Os})\\
\hspace{0.5cm}(\HOLConst{CFG} \\
\hspace{0.9cm}\HOLFreeVar{elementTest} \\
\hspace{0.9cm}\HOLFreeVar{stateInterp} \\
\hspace{0.9cm}\HOLFreeVar{context} \\
\hspace{0.9cm}(\HOLFreeVar{x}\HOLSymConst{::}\HOLFreeVar{ins}) \\
\hspace{0.9cm}\HOLFreeVar{s} \\
\hspace{0.9cm}\HOLFreeVar{outs})
\end{tabbing}
\parskip=18pt


\paragraph*{rule1}
\begin{tabbing}
\parskip=8pt
\HOLTokenTurnstile{} \\
\hspace{0.3cm}\HOLConst{TR} (\HOLFreeVar{M}\HOLSymConst{,}\HOLFreeVar{Oi}\HOLSymConst{,}\HOLFreeVar{Os}) (\HOLConst{trap} (\HOLConst{inputList} \HOLFreeVar{x}))\\
\hspace{0.5cm}(\HOLConst{CFG} \\
\hspace{0.9cm}\HOLFreeVar{elementTest} \\
\hspace{0.9cm}\HOLFreeVar{stateInterp} \\
\hspace{0.9cm}\HOLFreeVar{context} \\
\hspace{0.9cm}(\HOLFreeVar{x}\HOLSymConst{::}\HOLFreeVar{ins}) \\
\hspace{0.9cm}\HOLFreeVar{s} \\
\hspace{0.9cm}\HOLFreeVar{outs})\\
\hspace{0.5cm}(\HOLConst{CFG} \\
\hspace{0.9cm}\HOLFreeVar{elementTest} \\
\hspace{0.9cm}\HOLFreeVar{stateInterp} \\
\hspace{0.9cm}\HOLFreeVar{context} \\
\hspace{0.9cm}\HOLFreeVar{ins}\\
\hspace{0.9cm}(\HOLFreeVar{NS} \HOLFreeVar{s} (\HOLConst{trap} (\HOLConst{inputList} \HOLFreeVar{x})))\\
\hspace{0.9cm}(\HOLFreeVar{Out} \HOLFreeVar{s} (\HOLConst{trap} (\HOLConst{inputList} \HOLFreeVar{x}))\HOLSymConst{::}\HOLFreeVar{outs})) \\
\hspace{0.3cm}\HOLSymConst{\HOLTokenEquiv{}} \HOLConst{authenticationTest} \HOLFreeVar{elementTest} \HOLFreeVar{x} \\
\hspace{0.3cm}\HOLSymConst{\HOLTokenConj{}} \HOLConst{CFGInterpret} (\HOLFreeVar{M}\HOLSymConst{,}\HOLFreeVar{Oi}\HOLSymConst{,}\HOLFreeVar{Os})\\
\hspace{0.5cm}(\HOLConst{CFG} \\
\hspace{0.9cm}\HOLFreeVar{elementTest} \\
\hspace{0.9cm}\HOLFreeVar{stateInterp} \\
\hspace{0.9cm}\HOLFreeVar{context} \\
\hspace{0.9cm}(\HOLFreeVar{x}\HOLSymConst{::}\HOLFreeVar{ins}) \\
\hspace{0.9cm}\HOLFreeVar{s} \\
\hspace{0.9cm}\HOLFreeVar{outs})
\parskip=18pt
\end{tabbing}

\paragraph*{rule2}
\begin{tabbing}
\parskip=8pt
\HOLTokenTurnstile{} \\
\hspace{0.3cm}(\HOLConst{TR} (\HOLFreeVar{M}\HOLSymConst{,}\HOLFreeVar{Oi}\HOLSymConst{,}\HOLFreeVar{Os}) (\HOLConst{discard} (\HOLConst{inputList} \HOLFreeVar{x}))\\
\hspace{0.5cm}(\HOLConst{CFG} \\
\hspace{0.9cm}\HOLFreeVar{elementTest} \\
\hspace{0.9cm}\HOLFreeVar{stateInterp} \\
\hspace{0.9cm}\HOLFreeVar{context} \\
\hspace{0.9cm}(\HOLFreeVar{x}\HOLSymConst{::}\HOLFreeVar{ins}) \\
\hspace{0.9cm}\HOLFreeVar{s} \HOLFreeVar{outs})\\
\hspace{0.5cm}(\HOLConst{CFG} \\
\hspace{0.9cm}\HOLFreeVar{elementTest}\\
\hspace{0.9cm}\HOLFreeVar{stateInterp} \\
\hspace{0.9cm}\HOLFreeVar{context} \\
\hspace{0.9cm}\HOLFreeVar{ins}\\
\hspace{0.9cm}(\HOLFreeVar{NS} \HOLFreeVar{s} (\HOLConst{discard} (\HOLConst{inputList} \HOLFreeVar{x})))\\
\hspace{0.9cm}(\HOLFreeVar{Out} \HOLFreeVar{s} (\HOLConst{discard} (\HOLConst{inputList} \HOLFreeVar{x}))\HOLSymConst{::}\HOLFreeVar{outs})) \\
\hspace{0.3cm}\HOLSymConst{\HOLTokenEquiv{}}\\
\hspace{0.3cm}\HOLSymConst{\HOLTokenNeg{}}\HOLConst{authenticationTest} \HOLFreeVar{elementTest} \HOLFreeVar{x})
\parskip=18pt
\end{tabbing}


\subsection{Transition Definitions}


\paragraph*{(TR_rules, TR_cases_TR_ind)}
\begin{tabbing}
\parskip=8pt
\HOLTokenTurnstile{} \\
\hspace{0.3cm}(\HOLConst{TR} (\HOLFreeVar{M}\HOLSymConst{,}\HOLFreeVar{Oi}\HOLSymConst{,}\HOLFreeVar{Os}) (\HOLConst{exec} (\HOLConst{inputList} \HOLFreeVar{x}))\\
\hspace{0.5cm}(\HOLConst{CFG} \\
\hspace{0.9cm}\HOLFreeVar{elementTest} \\
\hspace{0.9cm}\HOLFreeVar{stateInterp} \\
\hspace{0.9cm}\HOLFreeVar{context} \\
\hspace{0.9cm}(\HOLFreeVar{x}\HOLSymConst{::}\HOLFreeVar{ins}) \\
\hspace{0.9cm}\HOLFreeVar{s} \\
\hspace{0.9cm}\HOLFreeVar{outs})\\
\hspace{0.5cm}(\HOLConst{CFG} \\
\hspace{0.9cm}\HOLFreeVar{elementTest} \\
\hspace{0.9cm}\HOLFreeVar{stateInterp} \\
\hspace{0.9cm}\HOLFreeVar{context} \\
\hspace{0.9cm}\HOLFreeVar{ins}\\
\hspace{0.9cm}(\HOLFreeVar{NS} \HOLFreeVar{s} (\HOLConst{exec} (\HOLConst{inputList} \HOLFreeVar{x})))\\
\hspace{0.9cm}(\HOLFreeVar{Out} \HOLFreeVar{s} (\HOLConst{exec} (\HOLConst{inputList} \HOLFreeVar{x}))\HOLSymConst{::}\HOLFreeVar{outs})) \\
\hspace{0.3cm}\HOLSymConst{\HOLTokenEquiv{}}\\
\hspace{0.3cm}\HOLConst{authenticationTest} \HOLFreeVar{elementTest} \HOLFreeVar{x} \HOLSymConst{\HOLTokenConj{}}\\
\hspace{0.3cm}\HOLConst{CFGInterpret} (\HOLFreeVar{M}\HOLSymConst{,}\HOLFreeVar{Oi}\HOLSymConst{,}\HOLFreeVar{Os})\\
\hspace{0.5cm}(\HOLConst{CFG} \\
\hspace{0.9cm}\HOLFreeVar{elementTest} \\
\hspace{0.9cm}\HOLFreeVar{stateInterp} \\
\hspace{0.9cm}\HOLFreeVar{context} (\HOLFreeVar{x}\HOLSymConst{::}\HOLFreeVar{ins}) \\
\hspace{0.9cm}\HOLFreeVar{s} \\
\hspace{0.9cm}\HOLFreeVar{outs})) \\
\hspace{0.3cm}\HOLSymConst{\HOLTokenConj{}}\\ 
\hspace{0.3cm}(\HOLConst{TR} (\HOLFreeVar{M}\HOLSymConst{,}\HOLFreeVar{Oi}\HOLSymConst{,}\HOLFreeVar{Os}) (\HOLConst{trap} (\HOLConst{inputList} \HOLFreeVar{x}))\\
\hspace{0.5cm}(\HOLConst{CFG} \\
\hspace{0.9cm}\HOLFreeVar{elementTest} \\
\hspace{0.9cm}\HOLFreeVar{stateInterp} \\
\hspace{0.9cm}\HOLFreeVar{context} \\
\hspace{0.9cm}(\HOLFreeVar{x}\HOLSymConst{::}\HOLFreeVar{ins}) \\
\hspace{0.9cm}\HOLFreeVar{s} \\
\hspace{0.9cm}\HOLFreeVar{outs})\\
\hspace{0.5cm}(\HOLConst{CFG} \\
\hspace{0.9cm}\HOLFreeVar{elementTest} \\
\hspace{0.9cm}\HOLFreeVar{stateInterp} \\
\hspace{0.9cm}\HOLFreeVar{context} \\
\hspace{0.9cm}\HOLFreeVar{ins}\\
\hspace{0.9cm}(\HOLFreeVar{NS} \HOLFreeVar{s} (\HOLConst{trap} (\HOLConst{inputList} \HOLFreeVar{x})))\\
\hspace{0.9cm}(\HOLFreeVar{Out} \HOLFreeVar{s} (\HOLConst{trap} (\HOLConst{inputList} \HOLFreeVar{x}))\HOLSymConst{::}\HOLFreeVar{outs})) \\
\hspace{0.3cm}\HOLSymConst{\HOLTokenEquiv{}}\\
\hspace{0.3cm}\HOLConst{authenticationTest} \HOLFreeVar{elementTest} \HOLFreeVar{x} \HOLSymConst{\HOLTokenConj{}}\\
\hspace{0.3cm}\HOLConst{CFGInterpret} (\HOLFreeVar{M}\HOLSymConst{,}\HOLFreeVar{Oi}\HOLSymConst{,}\HOLFreeVar{Os})\\
\hspace{0.5cm}(\HOLConst{CFG} \\
\hspace{0.9cm}\HOLFreeVar{elementTest} \\
\hspace{0.9cm}\HOLFreeVar{stateInterp} \\
\hspace{0.9cm}\HOLFreeVar{context} \\
\hspace{0.9cm}(\HOLFreeVar{x}\HOLSymConst{::}\HOLFreeVar{ins}) \\
\hspace{0.9cm}\HOLFreeVar{s} \HOLFreeVar{outs})) \\
\hspace{0.3cm}\HOLSymConst{\HOLTokenConj{}}\\
\hspace{0.3cm}(\HOLConst{TR} (\HOLFreeVar{M}\HOLSymConst{,}\HOLFreeVar{Oi}\HOLSymConst{,}\HOLFreeVar{Os}) (\HOLConst{discard} (\HOLConst{inputList} \HOLFreeVar{x}))\\
\hspace{0.5cm}(\HOLConst{CFG} \\
\hspace{0.9cm}\HOLFreeVar{elementTest} \\
\hspace{0.9cm}\HOLFreeVar{stateInterp} \\
\hspace{0.9cm}\HOLFreeVar{context} \\
\hspace{0.9cm}(\HOLFreeVar{x}\HOLSymConst{::}\HOLFreeVar{ins}) \\
\hspace{0.9cm}\HOLFreeVar{s} \HOLFreeVar{outs})\\
\hspace{0.5cm}(\HOLConst{CFG} \\
\hspace{0.9cm}\HOLFreeVar{elementTest}\\
\hspace{0.9cm}\HOLFreeVar{stateInterp} \\
\hspace{0.9cm}\HOLFreeVar{context} \\
\hspace{0.9cm}\HOLFreeVar{ins}\\
\hspace{0.9cm}(\HOLFreeVar{NS} \HOLFreeVar{s} (\HOLConst{discard} (\HOLConst{inputList} \HOLFreeVar{x})))\\
\hspace{0.9cm}(\HOLFreeVar{Out} \HOLFreeVar{s} (\HOLConst{discard} (\HOLConst{inputList} \HOLFreeVar{x}))\HOLSymConst{::}\HOLFreeVar{outs})) \\
\hspace{0.3cm}\HOLSymConst{\HOLTokenEquiv{}}\\
\hspace{0.3cm}\HOLSymConst{\HOLTokenNeg{}}\HOLConst{authenticationTest} \HOLFreeVar{elementTest} \HOLFreeVar{x})
\parskip=18pt
\end{tabbing}



\paragraph*{TR_exec_cmd_rule}
\begin{tabbing}
\parskip=8pt
\HOLTokenTurnstile{} \\
\hspace{0.3cm}\HOLSymConst{\HOLTokenForall{}}\HOLBoundVar{elementTest} \HOLBoundVar{context} \HOLBoundVar{stateInterp} \HOLBoundVar{x} \HOLBoundVar{ins} \HOLBoundVar{s} \HOLBoundVar{outs}.\\
\hspace{0.5cm}(\HOLSymConst{\HOLTokenForall{}}\HOLBoundVar{M} \HOLBoundVar{Oi} \HOLBoundVar{Os}.\\
\hspace{0.9cm}\HOLConst{CFGInterpret} (\HOLBoundVar{M}\HOLSymConst{,}\HOLBoundVar{Oi}\HOLSymConst{,}\HOLBoundVar{Os})\\
\hspace{1.1cm}(\HOLConst{CFG} \\
\hspace{1.3cm}\HOLBoundVar{elementTest} \\
\hspace{1.3cm}\HOLBoundVar{stateInterp} \\
\hspace{1.3cm}\HOLBoundVar{context} \\
\hspace{1.3cm}(\HOLBoundVar{x}\HOLSymConst{::}\HOLBoundVar{ins}) \\
\hspace{1.3cm}\HOLBoundVar{s}\\
\hspace{1.3cm}\HOLBoundVar{outs}) \\
\hspace{0.9cm}\HOLSymConst{\HOLTokenImp{}}(\HOLBoundVar{M}\HOLSymConst{,}\HOLBoundVar{Oi}\HOLSymConst{,}\HOLBoundVar{Os}) \HOLConst{satList} \HOLConst{propCommandList} \HOLBoundVar{x}) \\
\hspace{0.5cm}\HOLSymConst{\HOLTokenImp{}} \HOLSymConst{\HOLTokenForall{}}\HOLBoundVar{NS} \HOLBoundVar{Out} \HOLBoundVar{M} \HOLBoundVar{Oi} \HOLBoundVar{Os}.\\
\hspace{0.9cm}\HOLConst{TR} (\HOLBoundVar{M}\HOLSymConst{,}\HOLBoundVar{Oi}\HOLSymConst{,}\HOLBoundVar{Os}) (\HOLConst{exec} (\HOLConst{inputList} \HOLBoundVar{x}))\\
\hspace{1.1cm}(\HOLConst{CFG} \\
\hspace{1.3cm}\HOLBoundVar{elementTest} \\
\hspace{1.3cm}\HOLBoundVar{stateInterp} \\
\hspace{1.3cm}\HOLBoundVar{context} \\
\hspace{1.3cm}(\HOLBoundVar{x}\HOLSymConst{::}\HOLBoundVar{ins}) \\
\hspace{1.3cm}\HOLBoundVar{s} \\
\hspace{1.3cm}\HOLBoundVar{outs})\\
\hspace{1.1cm}(\HOLConst{CFG} \\
\hspace{1.3cm}\HOLBoundVar{elementTest} \\
\hspace{1.3cm}\HOLBoundVar{stateInterp} \\
\hspace{1.3cm}\HOLBoundVar{context} \\
\hspace{1.3cm}\HOLBoundVar{ins}\\
\hspace{1.3cm}(\HOLBoundVar{NS} \HOLBoundVar{s} (\HOLConst{exec} (\HOLConst{inputList} \HOLBoundVar{x})))\\
\hspace{1.3cm}(\HOLBoundVar{Out} \HOLBoundVar{s} (\HOLConst{exec} (\HOLConst{inputList} \HOLBoundVar{x}))\HOLSymConst{::}\HOLBoundVar{outs})) \\
\hspace{0.5cm}\HOLSymConst{\HOLTokenEquiv{}} \HOLConst{authenticationTest} \HOLBoundVar{elementTest} \HOLBoundVar{x} \\
\hspace{0.5cm}\HOLSymConst{\HOLTokenConj{}} \HOLConst{CFGInterpret} (\HOLBoundVar{M}\HOLSymConst{,}\HOLBoundVar{Oi}\HOLSymConst{,}\HOLBoundVar{Os})\\
\hspace{0.9cm}(\HOLConst{CFG} \\
\hspace{1.1cm}\HOLBoundVar{elementTest} \\
\hspace{1.1cm}\HOLBoundVar{stateInterp} \\
\hspace{1.1cm}\HOLBoundVar{context} \\
\hspace{1.1cm}(\HOLBoundVar{x}\HOLSymConst{::}\HOLBoundVar{ins}) \\
\hspace{1.1cm}\HOLBoundVar{s} \\
\hspace{1.1cm}\HOLBoundVar{outs}) \\
\hspace{0.5cm}\HOLSymConst{\HOLTokenConj{}} (\HOLBoundVar{M}\HOLSymConst{,}\HOLBoundVar{Oi}\HOLSymConst{,}\HOLBoundVar{Os}) \HOLConst{satList} \HOLConst{propCommandList} \HOLBoundVar{x}
\parskip=18pt
\end{tabbing}


\paragraph*{TR_discard_cmd_rule}
\begin{tabbing}
\parskip=8pt
\HOLTokenTurnstile{}\\
\hspace{0.3cm}\HOLConst{TR} (\HOLFreeVar{M}\HOLSymConst{,}\HOLFreeVar{Oi}\HOLSymConst{,}\HOLFreeVar{Os}) (\HOLConst{discard} (\HOLConst{inputList} \HOLFreeVar{x}))\\
\hspace{0.5cm}(\HOLConst{CFG}\\
\hspace{0.9cm}\HOLFreeVar{elementTest}\\ 
\hspace{0.9cm}\HOLFreeVar{stateInterp}\\
\hspace{0.9cm}\HOLFreeVar{context}\\ 
\hspace{0.9cm}(\HOLFreeVar{x}\HOLSymConst{::}\HOLFreeVar{ins})\\
\hspace{0.9cm} \HOLFreeVar{s}\\ 
\hspace{0.9cm}\HOLFreeVar{outs})\\
\hspace{0.5cm}(\HOLConst{CFG}\\ 
\hspace{0.9cm}\HOLFreeVar{elementTest} \\
\hspace{0.9cm}\HOLFreeVar{stateInterp} \\
\hspace{0.9cm}\HOLFreeVar{context} \\
\hspace{0.9cm}\HOLFreeVar{ins}\\
\hspace{0.9cm}(\HOLFreeVar{NS} \HOLFreeVar{s} (\HOLConst{discard} (\HOLConst{inputList} \HOLFreeVar{x})))\\
\hspace{0.9cm}(\HOLFreeVar{Out} \HOLFreeVar{s} (\HOLConst{discard} (\HOLConst{inputList} \HOLFreeVar{x}))\HOLSymConst{::}\HOLFreeVar{outs})) \\
\hspace{0.3cm}\HOLSymConst{\HOLTokenEquiv{}}
   \HOLSymConst{\HOLTokenNeg{}}\HOLConst{authenticationTest} \HOLFreeVar{elementTest} \HOLFreeVar{x}
\parskip=18pt
\end{tabbing}




\end{document}