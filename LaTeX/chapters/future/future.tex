%future
\documentclass[../../main/main.tex]{subfiles}

\begin{document}
\title{Future Work}
%%%%%%%%%%%%%%%%%%%%% Chapter Future Work%%%%%%%%%%%%%%
\chapter{Future Work}\label{chp:future}

This chapter discusses plans and possibilities for future work with the patrol base operations and \gls{storm}.


\section{Patrol Base Operations}\label{sec:futureops}
The purpose of this master thesis is to demonstrate \gls{storm} on a non-automated, human-centered system.  This work is complete.  There are currently no plans to extend this particular project.  However, this project left the author with a mindful of ideas on how to modify the secure state machines to manage the variety and complexity of the patrol base operations.   Some modifications are discussed in the previous chapter.  It would be interesting to see how far \gls{storm} could go with regards to the patrol base operations and military operations in general.  For example, would \gls{storm} be effective on a battalion, mission, entire operations, and the U.S. Army as  a whole?


\section{Automation}\label{sec:futureautomation}
There is no doubt that the future of non-automated, human-centered systems is, at least, partial automation.  Partial automation could be an accountability system that tracks soldiers through the operations using an implant, chip embedded in dog tags, a hand-held device, etc..  It could be an accountability system that tracks equipment (an equipment module). It could also be an accountability system that tracks the progression of operations as a whole by combining tracking data from one system with tracking data and intelligence from other systems.\footnote{Artificial intelligence could be used to analyze the operations as a whole, make predictions and simulations, and assist decision-makers at the Pentagon, for example, in determining the best course of action for the operations.  It could also be used to review and analyze the operations after the fact to make improvements in future operations.}  It could even be an accountability system that tracks the health of soldiers.\footnote{There already exists micro-bots that track blood pressure and whether an individual has taken his medication.  This data is transmitted via wifi to the appropriate medical practitioners. }  With advances in computational power, automated intelligent systems, and technology nearly anything imaginable is possible. 

The design of the patrol base operations as a modularized hierarchy of secure state machines is readily automated. The automation follows from the system's well-defined behavior, which is constrained by the nature of the secure state machines.  Furthermore, secure state machines enforce security as demonstrated by the formal application of the access-control logic.

      %%%%%%%%%%%%%%%%%%% Section Accountability Systems %%%%%%%%      
\subsection{STORM}\label{sec:futurestorm}
The components of \gls{storm} are gaining popularity because they are proving effective and efficient for analyzing system safety and security.  Already, additional components for \gls{stpa} exist to analyze other systems engineering concerns such as privacy and large, software intensive systems.  Such systems may also be \gls{storm}-applicable, particularly systems where privacy is a concern such as medical records or communications systems.  \gls{stpasec} could be supplemented with \gls{stpa}-Priv.   \gls{csbd} is already designed to reason about privacy.

Future work for \gls{storm} involves generating more examples of \gls{storm} and training individuals to use it.  There is also an application for the \gls{stpasec} component of \gls{storm} called XSTAMPP.  But, there remains some bugs to work through.\footnote{Some people have got the program to work.}




What is unquestionable is the need to develop systems that are trustworthy.   Trustworthiness guarantees the safety, security, and privacy of man-made systems and can only be realized through rigorous analysis of best practices in the field of systems security engineering.  The  field's Systems Security Engineering Framework guides a security analysis and is implemented through methodologies such as  \gls{storm}.  \gls{storm} works.  It is effective and efficient, and that's the bottom line.


%The applicability of \glsentryshort{csbd} has already been demonstrated with automated systems.  With this master thesis, it has been demonstrated on two extremes of the range of man-made systems.  It is not too much of a leap to apply \glsentryshort{csbd} to a mixture of the two.  

%An idea that came up often during this work is the use of \glsentryshort{csbd} to designing accountability systems. For example, soldiers could have an application where they scan their equipment, enter their state of health, and provide other information.  The application would then send a message to the platoon computer (or head quarters or the pentagon) that the soldier is battle ready.  This requires access-control.  Essentially, the application speaks for the soldier.  The applications says \textit{Application "quoting" SoldierGIJane says battleReady}.  A policy would then include information describing the conditions whereby \textit{SoldierGIJane controls battleReady}. 
%
%Furthermore, the application could track phases of the patrol base operations.  This information could track personnel and equipment.   Or, the state of the operations could be fed into a machine learning program that calculates strategies in combination with information from other operations- (a real-time mission analysis).   This could be done using a variety of signals from the patrol base operations to indicate operational phase. For example, radio confirmation indicating the mission is received would place the operations in the planning phase.  The platoon leader could tap a link on his phone to indicate movement to the objective rally point.  And so on.  Such an accountability system will require some form of access control to verify that the right messages are coming from the right people.  The method discussed in this master thesis is an effective way to do this.

\end{document}