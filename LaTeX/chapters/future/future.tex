%future
\documentclass[../../main/main.tex]{subfiles}

\begin{document}
\title{Future Work}
%%%%%%%%%%%%%%%%%%%%% Chapter Future Work%%%%%%%%%%%%%%
\chapter{Future Work}\label{chp:future}
The previous chapters discussed the application of \glsentryshort{csbd} to a patrol base operations,  This chapter touches briefly an other hon-automated human-centered systems where this work is applicable. It also touches briefly on the applicability of \glsentryshort{csbd} to a accountability systems.

\section{Applicability}\label{sec:accountability}


There are numerous non-automated, human-centered systems that require some form of security.  For example, a building evacuation plan requires clear delineation of who controls a return to the building.  Can anyone say "its safe to return to the building?"  Or, must if be a fire fighter or police officer?  On a larger scale, national disaster emergency management requires planning for leadership, acquisition of resources, and communication with the media.  If a police officer calls the team leader and says he needs four more fire engines, should that request be granted?  Does the police officer have the authority?  These are issues that should be worked out during the disaster response planning phase.  

The work in this master thesis demonstrates that a large non-automated, human-centered system can be designed with access-control in mind.  But, this means that it can also be applied to smaller systems such as active-shooter responses.  The use of a hierarchy of secure state machines need not be implemented for every phase of the system, but this design strategy allows the engineer to see the system in more detail.  This detail can help highlight areas where access control is needed.  With or without the formal verification and documentation, this approach is useful to designing more trustworthy systems with regards to access control.


      %%%%%%%%%%%%%%%%%%% Section Accountability Systems %%%%%%%%
      
\subsection{Accountability Systems}\label{sec:accountability}
The applicability of \glsentryshort{csbd} has already been demonstrated with automated systems.  With this master thesis, it has been demonstrated on two extremes of the range of man-made systems.  It is not too much of a leap to apply \glsentryshort{csbd} to a mixture of the two.  

An idea that came up often during this work is the use of \glsentryshort{csbd} to designing accountability systems. For example, soldiers could have an application where they scan their equipment, enter their state of health, and provide other information.  The application would then send a message to the platoon computer (or head quarters or the pentagon) that the soldier is battle ready.  This requires access-control.  Essentially, the application speaks for the soldier.  The applications says \textit{Application "quoting" SoldierGIJane says battleReady} .  A policy would then include information describing the conditions whereby \textit{SoldierGIJane controls battleReady}. 

Furthermore, the application could track phases of the patrol base operations.  This information could track personnel and equipment.   Or, the state of the operations could be fed into a machine learning program that calculates strategies in combination with information from other operations- (a real-time mission analysis).   This could be done using a variety of signals from the patrol base operations to indicate operational phase. For example, radio confirmation indicating the mission is received would place the operations in the planning phase.  The platoon leader could tap a link on his phone to indicate movement to the objective rally point.  And so on.  Such an accountability system will require some form of access control to verify that the right messages are coming from the right people.  The method discussed in this master thesis is an effective way to do this.

\end{document}