%future
\documentclass[../../main/main.tex]{subfiles}

\begin{document}
\title{Future Work}
%%%%%%%%%%%%%%%%%%%%% Chapter Future Work%%%%%%%%%%%%%%
\chapter{Future Work}\label{chp:future}

      %%%%%%%%%%%%%%%%%%% Section Accountability Systems %%%%%%%%%
\section{Applicability}\label{sec:accountability}

Emergency Management systems require human communications.  As with most systems of communication, the communicators need to work with the confidence that they are receiving instructions from the right source with the right authority.  A remodeling of the system with complete mediation in mind.  From the work in this thesis, the system could simply be modeled in a way that highlights communication avenues and applies CSBD to verify complete mediation.

On perhaps a smaller scale, but no less significant, are non-automated, human-centered school drills such as fire drills and active shooter drills.  Clearly, a systems engineering approach could be used to model these drills.  The systems security engineering aspect would cover things such as verifying that safety zones are assigned and reasonably accessible, "what to do if" scenarios problems are discussed and solved, and who to trust under what circumstances are solved.  Complete mediation would be intertwined within this system design.  For example, when should students leave their classrooms after lock-down?  Who says so?  How do you know if that person is trust worthy?  If its the teacher or a police officer, then authentication is typically a matter of visual recognition as it is with authentication in the patrol base operations.  Regardless of how these are decided, these things should be decided during the design phase of the drill and the concepts of complete mediation should be built-in to the design.

      
\subsection{Accountability Systems}\label{sec:accountability}


\end{document}